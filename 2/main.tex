\documentclass{article}
\usepackage{amsmath}

\begin{document}

	\title{Формальная постановка задачи оптимизации расписания с использованием алгоритма имитации отжига}
	\author{Санников Тимофей Владимирович}
	\date{}
	\maketitle
	
	\section*{Дано:}
	
	\begin{itemize}
	    \item \( N \) — количество независимых работ.
	    \item \( M \) — количество процессоров.
	    \item \( t_i \), \( i = 1, 2, \dots, N \) — время выполнения каждой работы \( i \), \( t_i > 0 \).
	\end{itemize}
	
	\section*{Требуется:}
	
	Построить расписание выполнения всех
	\( N \) работ на \( M \) процессорах без прерываний,
	чтобы минимизировать один из следующих критериев:
	
	\section*{Минимизируемый критерий:}
	
		
	Критерий выбирается на основе значения контрольной суммы CRC32 от фамилии и инициалов:
	\begin{itemize}
	    \item Остаток от деления CRC32 на 2 равен 1: минимизируется \( K_1 \).
	    \item Остаток от деления CRC32 на 2 равен 0: минимизируется \( K_2 \).
	\end{itemize}
	CRC32 для моего имени и инициалов = 2613818349, что означает, что мой критерий -- К1
	

	Критерий \( K_1 \): Разбалансированность расписания
	\[
	K_1 = T_{\text{max}} - T_{\text{min}},
	\]
	где:
	\[
	T_{\text{max}} = \max_{j=1, 2, \dots, M} T_j,
	\]
	\[
	T_{\text{min}} = \min_{j=1, 2, \dots, M} T_j,
	\]
	а \( T_j \) — это время завершения всех назначенных на процессор \( j \) работ:
	\[
	T_j = \sum_{i \in \mathcal{J}_j} t_i.
	\]
	\section*{Дополнительные условия:}
	
	Расписание должно удовлетворять следующим ограничениям:
	\begin{itemize}
	    \item Каждая работа выполняется на одном процессоре без прерываний.
	    \item Одна работа не может быть запущена на нескольких процессорах одновременно.
	    \item Время выполнения каждой работы \( t_i \) фиксировано.
	\end{itemize}

\end{document}

